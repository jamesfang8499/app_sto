\documentclass{ctexart}
\usepackage[top=1cm, bottom=2.5cm, left=2.5cm, right=2.5cm]{geometry}
\usepackage{hyperref}

\begin{document}
\fontsize{14}{16}\selectfont

\title{关于课件源代码的几点说明}
\date{}
\maketitle
\thispagestyle{empty}
需要对课件内容进行修改和完善的老师,可以使用\LaTeX{}对源文件进行操作。以下几点是使用的注意事项:
\begin{enumerate}
\item tex文件是课件的源代码,文件夹fig存放的是对应的图片,请不要更改文件夹与源代码的相对路径关系,否则编译会报错。
\item 	建议使用TeX Live2019及以上版本的\LaTeX{}发行版进行代码的编译。如果使用CTEX中文套装会出现难以解决的编译错误和兼容性问题。使用苹果Mac系统的老师,可以使用MacTeX。
\item 	由于代码当中调用了系统字体和多种图片格式,因此编译方式请采用Xe\LaTeX{},其他的编译方式会报错。
\item 确保使用的编辑器支持UTF8编码,推荐使用TeX Live默认的编辑器TeXworks或优秀的第三方开发环境TeXStudio、VS Code。使用苹果Mac系统的老师,可以使用TeXShop或Mac版TeXStudio。
\item 	由于源代码当中包含了交叉引用的内容,首次编译需要运行两次Xe\LaTeX{},否则会出现目录空白和公式编号引用显示为问号的问题。
\item 	TeX Live可通过\url{www.tug.org}网站免费下载,推荐使用迅雷等下载工具,通过torrent种子文件进行ISO格式安装文件的下载;TeX Live的安装可参考B站(\url{www.bilibili.com})等网站的视频资源。使用苹果Mac系统的老师,可以在\url{www.tug.org}下载并安装MacTeX,用法与TeX Live完全一致。
\item 	由于课件中的中文使用的是微软雅黑字体,请勿将其用于商业用途,以免引起法律纠纷。可以通过\verb|\setCJKmainfont|命令重新指定其他本地安装的开源字体以避免此类版权纠纷。
\item 与源代码编译有关的问题,也可通过以下邮箱与我们联系:

\verb|james_fang_fe2016@163.com|
\end{enumerate}
















\end{document}